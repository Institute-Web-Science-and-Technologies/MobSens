\documentclass[a4paper]{scrreprt}
 
\usepackage{enumitem}
\usepackage[german]{babel}
\usepackage[utf8]{inputenc}
\usepackage[T1]{fontenc}
\usepackage{ae}
\usepackage[bookmarks,bookmarksnumbered]{hyperref}

\title{Mobile Sensors\\Klassifikation}

\begin{document}

\maketitle

\tableofcontents

\chapter{Einleitung}
Ziel von Mobile Sensors ist es eine mit dem Fahrad gefahrene Strecke über von Smartphones gesammelten Sensordaten zu bewerten.
Dazu sollen die anfallenden Daten mit Techniken und Methoden des maschinellen Lernens ausgewertet und klassifziert werden.

Da die Masse der pro Strecke anfallenden Sensordaten zu groß und inhomogen ist, um eine direkte Bewertung ableiten zu können, muss sie zunächst in verarbeitbare Einheiten zerlegt werden. 
Dabei wird eine gefahrene Strecke als eine Abfolge von Ereignissen (Bremsen, Ausweichen, ...) interpretiert, die Starke Ausschläge der Sensoren zur Folge haben und so klassifiziert werden können.
Diese Serialisierung einer Serialisierung einer Strecke kann dann wiederum einer Klassifikation unterzogen werden.

\chapter{Ereignis Klassifikation}
Im Folgenden wird beschrieben, wie Ereignisse mittels \textit{Supervised Learning}, einer Technik aus dem Bereich des machinellen Lernens erkannt werden sollen. Beim \textit{Supervised Learning} wird algorithmisch anhand von bereits erkannten Ereignisklassen (Trainingsdaten) eine allgemeinre Regel abgeleitet, die über neue Daten entscheidet, welcher Ereignisklasse sie zugehören.
\section{Zu klassifizierende Ereignisse}
Zunächst werden die Ereignisse/Klassen genauer umrissen, welche hoffentlich anhand der Sensordaten unterscheidbar sind.

\subsection{Nicht-Ereigis/Grundrauschen}
Um Ereignisse als Abweichung eines Normalzustandes erkennen zu können, muss dieser zuerst klassifiziert werden.
Eine Strecke wird physikalisch gesehen nie hundertprozentig ereignisfrei sein, da immer minimale Erschütterungen oder minimale Beschleunigungs- bzw. Bremsvorgänge auftreten.
Diese sind allerdings nicht so signifikant, dass sie eine Fahrtbeeintrechtigung darstellen.
Genauso gehört ein sanftes Abbiegen in einer Kurve zum normalen Fahrverhalten.
Solche Messungen können daher als eine Art Grundrauschen betrachtet werden.

\subsection{Abruptes Bremsen/Beschleunigen}
Ein abruptes Bremsen oder gar Anhalten aufgrund von Hindernissen bzw. hoher Verkehrsdichte ist eine offensichtliche Beeinträchtigung einer Fahrt.
Folgen auf solche Bremsvorgänge direkt eine ebenso abrupte Beschleunigungen (\textit{Stop-and-Go}), indiziert dies, dass das Bremens ungewollt war, also eine Beeinträchtigung der Fahrt.

\subsection{Ausweichen}
Ausweichbewegungen deuten meist auch auf Hindernisse hin und sind somit auch ein Indikator für den Verkehrsfluss. 
Zusätzlich zu diesen Events können weitere Berechnungen auf den Daten getätigt werden um noch mehr Informationsgehalt für eine Streckenklassifizierung zu tätigen. Beispielsweise wäre dies die Durchschnittsgeschwindigkeit.
Der Hauptaspekt für das Projekt liegt jedoch darin die oben genannten Events in einer Aufzeichnung zu erkennen und dem Benutzer darzustellen. Die Gesamtklassifikation eine Strecke und das Hinzufügen von weiteren Events bietet eine optionale Erweiterung. Weitere optionale Informationen wären beispielsweise Straßenbelag oder Temperatur/Wetter.

\subsection{Erschütterung durch Hindernis}
Eine Erschütterung stellt für einen Fahrradfahrer eine Behinderung, bzw. eine Störung des Verkehrsflusses dar. 
Erschütterungen werden meist durch kleine Hindernisse verursacht, denen nicht ausgewichen wird, z.B.: Borsteine, Schlaglöcher, Kopfsteinpflaster, etc.
\section{Sammeln von Trainingsdaten}
Hier soll ein Konzept beschrieben werden, wie systematisch eine Datengrundlage erstellt werden kann, um die bereits genannten Ereignisse zu klassifizieren.

\subsection{Lagerung des Smartphones}
Die Lagerung des Smartphones ist entscheidend für die Ausschlagstärke eines Ereignisses. 
Befindet sich das Smartphone in einer statischen Halterung am Lenker, werden die physikalischen Kräfte fast ungedämpft auf die Sensoren übertragen.
Befindet sich das Smartphone im Gegensatz dazu in der Hosentasche des Fahrers, wirken die Ereignisse weniger stark auf die Sensoren ein und werden zudem noch von Trittbewegungen überlagert.
Beide Varianten sind zu testen.

Zu Beachten ist die Ausrichtung des Smartphones in der Hosentasche. Beschleunigungsmessungen haben je nach Ausrichtung ähnliche Ausschläge auf verschiedenen Raumachsen.

\subsection{Untergrund}
Erschütterungen hängen nicht nur von der Lagerung des Smartphones sondern maßgeblich vom Untergrund der Fahrt ab.
Witterung und Art des Untergrunds verändern Auschläge unter Umständen zwar nur marginal.
Dies kann allerdings zu falschen Klassifikationen führen, falls die Varianz der Trainingsdaten zu gering ist.
Mögliche Untergründe sind:
\begin{itemize}
\item Asphalt
\item Plfaster
\item Schotter
\item Wiese
\end{itemize}
10 Aufnahmen a 100m je Ereigniss sind zu testen.

\subsection{Ereigniss Ausprägung}
Ereignisse können in ihrer Ausprägung variieren. 
Bei zu homogenen Testfahrten birgt dies die Gefahr der späteren Misklassfikation.
Bei mehreren Testfahrten ist darauf zu achten, dass sie sich für die verschiedenen Ereignisse wie folgt unterscheiden:
\begin{itemize}
\item Gerade aus fahren in unterschiedlichen Geschwindigkeiten
\item Kurven fahren in verschieden starken Neigungen
\item Bremsen/Beschleunigen in wechselnder Intensität
\item Ausweichen in rechts-links und links-rechts
\item Bordsteine herauf und herab 
\end{itemize}



\section{Auswertung}
\subsection{Auswertung der Traingsdaten}
\subsubsection{Nicht-Ereigis/Grundrauschen}


\subsubsection{Abruptes Bremsen/Beschleunigen}
\textbf{Vermutung:}
Accelerometerdaten zeigen starke Ausschläge und lassen sich mittels Thresholds transformieren.
\newline
\textbf{Sensoren:}
Accelerometer / Linear Acceleration
\subsubsection{Ausweichen}
\textbf{Vermutung:}
Nach einem Ausschlag in eine Richtung muss ein negativer Ausschlag in Gegenrichtung vorkommen.
\newline
\textbf{Sensoren:}
Accelerometer / Gyroskop

\subsubsection{Erschütterung durch Hindernis}
\textbf{Vermutung:}
Hoher Bordstein
Starker Ausschlag in eine Richtung
\newline
\textbf{Sensoren:}
Accelerometer / Gravity

\subsection{Auswahl geeigneter Klassifikationstechniken}

\chapter{Strecken Klassifikation}


\end{document}