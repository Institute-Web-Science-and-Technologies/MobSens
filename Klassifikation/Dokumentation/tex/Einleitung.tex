\chapter{Einleitung}
Ziel von Mobile Sensors ist es eine mit dem Fahrad gefahrene Strecke über von Smartphones gesammelten Sensordaten zu bewerten.
Dazu sollen die anfallenden Daten mit Techniken und Methoden des maschinellen Lernens ausgewertet und klassifziert werden.

Da die Masse der pro Strecke anfallenden Sensordaten zu groß und inhomogen ist, um eine direkte Bewertung ableiten zu können, muss sie zunächst in verarbeitbare Einheiten zerlegt werden. 
Dabei wird eine gefahrene Strecke als eine Abfolge von Ereignissen (Bremsen, Ausweichen, ...) interpretiert, die Starke Ausschläge der Sensoren zur Folge haben und so klassifiziert werden können.
Diese Serialisierung einer Serialisierung einer Strecke kann dann wiederum einer Klassifikation unterzogen werden.