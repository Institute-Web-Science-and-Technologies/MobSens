\section{Sammeln von Trainingsdaten}
Hier soll ein Konzept beschrieben werden, wie systematisch eine Datengrundlage erstellt werden kann, um die bereits genannten Ereignisse zu klassifizieren.

\subsection{Lagerung des Smartphones}
Die Lagerung des Smartphones ist entscheidend für die Ausschlagstärke eines Ereignisses. 
Befindet sich das Smartphone in einer statischen Halterung am Lenker, werden die physikalischen Kräfte fast ungedämpft auf die Sensoren übertragen.
Befindet sich das Smartphone im Gegensatz dazu in der Hosentasche des Fahrers, wirken die Ereignisse weniger stark auf die Sensoren ein und werden zudem noch von Trittbewegungen überlagert.
Beide Varianten sind zu testen.

Zu Beachten ist die Ausrichtung des Smartphones in der Hosentasche. Beschleunigungsmessungen haben je nach Ausrichtung ähnliche Ausschläge auf verschiedenen Raumachsen.

\subsection{Untergrund}
Erschütterungen hängen nicht nur von der Lagerung des Smartphones sondern maßgeblich vom Untergrund der Fahrt ab.
Witterung und Art des Untergrunds verändern Auschläge unter Umständen zwar nur marginal.
Dies kann allerdings zu falschen Klassifikationen führen, falls die Varianz der Trainingsdaten zu gering ist.
Mögliche Untergründe sind:
\begin{itemize}
\item Asphalt
\item Plfaster
\item Schotter
\item Wiese
\end{itemize}
10 Aufnahmen a 100m je Ereigniss sind zu testen.

\subsection{Ereigniss Ausprägung}
Ereignisse können in ihrer Ausprägung variieren. 
Bei zu homogenen Testfahrten birgt dies die Gefahr der späteren Misklassfikation.
Bei mehreren Testfahrten ist darauf zu achten, dass sie sich für die verschiedenen Ereignisse wie folgt unterscheiden:
\begin{itemize}
\item Gerade aus fahren in unterschiedlichen Geschwindigkeiten
\item Kurven fahren in verschieden starken Neigungen
\item Bremsen/Beschleunigen in wechselnder Intensität
\item Ausweichen in rechts-links und links-rechts
\item Bordsteine herauf und herab 
\end{itemize}
